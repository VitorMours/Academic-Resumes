\documentclass[12pt a4paper]{paper}
\usepackage{array}

\begin{document} 
\section{Protocolos e Modelos} % (fold)
\label{sec:Protocolos e Modelos}
\subsection{Fundamentos da Comunicação} % (fold)
\label{sub:Fundamentos da Comunicação}
As redes variam em tamanho, forma e função. Elas podem ser tão complexas quanto os 
dispositivos conectados, ou tão simples quanto dois computadores conectados diretamente
um ao outro com um único cabo e qualquer outra coisa.
As pessoas trocam ideias usando vários métodos de comunicação diferentes. No entanto, 
todos os métodos de comunicação têm os seguintes três elementos em comum:

\begin{itemize}
  \item \textbf{Fonte da Mensagem: } As fontes da mensagem são pessoas ou dispositivos eletrônicos que precisam enviar uma mensagem para outro lugar.
  \item \textbf{Destino da Mensagem: }O destino recebe a mensaem e a interpreta.
  \item \textbf{Canal: }Consiste na mídia que fornece o caminho pelo qual a mensagem viaja da oridem ao destino.
\end{itemize}
% subsection Fundamentos da Comunicação (end)

\subsection{Protocolos de Comunicação} % (fold)
\label{sub:Protocolos de Comunicação}
O envio da mensagem, é regido por regras que são chamadas de protocolos. 
Esses protocolos são específicos ao tipo de método de comunicação que estão sendo usado.
Em nossa comunicação pessoal do dia-a-dia, as regras que usamos para nos comunicar 
em uma mídia.
% subsection Protocolos de Comunicação (end)

\subsection{Requisitos de protocolo de rede} % (fold)
\label{sub:Requisitos de protocolo de rede}
Os protocolos usado nas comunicações de rede compartilham muitas dessas características
fundamentais. Além de identificar a origem e o destino, os protocolos de computadores e 
de redes definem os detalhes sobre como uma mensagem é trnasmitida por uma rede. 
Eles possuem os seguintes requisitos:

\begin{itemize}
  \item Codificação de mensagens.
  \item Formatação e encapsulamento de mensagens. 
  \item Tamanho da mensagem. 
  \item Tempo da mensagem. 
  \item Opção de envio de mensagem.
\end{itemize}
% subsection Requisitos de protocolo de rede (end)

\subsection{Codificação de Mensagens} % (fold)
\label{sub:Codificação de Mensagens}
Uma das primeiras etapas para enviar uma mensagem, é codificar ela. A codificação é o 
processo de conversão de informações em outra forma aceitável para a transmissão. A 
decodificação reverte esse processo para interpretar as informações.
% subsection Codificação de Mensagens (end)

\subsection{Formatação e Encapsulamento de Mensagens} % (fold)
\label{sub:Formatação e Encapsulamento de Mensagens}
Quando uma mensagem é enviada da origem para o destino, deve usar um formato e 
estrutura específica, para que seu uso seja o mais eficiente possível. Dentro da rede, 
isso se transforma com o uso de IP, como uma função de localização para identificar a 
origem e o destino da mensagem.
% subsection Formatação e Encapsulamento de Mensagens (end)

\subsection{Tamanho da Mensagem} % (fold)
\label{sub:Tamanho da Mensagem}
Quando uma mensagem é muita longa, para ela ser enviada de um host para o outro, é 
necessário dividir a mensagem em partes menores, pois isso facilita o transporte e a 
detecção de possíveis falhas dentro da nossa comunicação




% subsection Tamanho da Mensagem (end)

\subsection{Temporização da Mensagem} % (fold)
\label{sub:Temporização da Mensagem}
O tempo que as mensagens possuem de vida também é extremamente importante, e isso é 
controlado por meio de três elementos que estão presente, sendo eles: 

\begin{itemize}
  \item \textbf{Controle de Fluxo: } É o processo de gerenciamento  da taxa de transmissão dos dados. o controle define quanta informação pode ser enviada e a velocidade com que pode ser entrege.
  \item \textbf{Tempo limite de resposta: } Se uma pessoa fizer uma pergunta e nçao ouvir uma resposta dentro de um período aceitável de tempo, entende que a resposta nunca deve chegar, o mesmo funciona para os computadores. Os hosts da rede usam protocolos de rede e especificam quanto tempo de espera pode ser ofertado para que uma ação seja executada.
  \item \textbf{Método de Acesso: } Ajuda a determinar quando alguém pode enviar uma mensagem. Quando um dispositivo deseja transmitir uma mensagem, sendo por fio ou sem fio, deve existir um tipo de conecção, podendo ser ela por LAN ou WLAN, respectivamente.
\end{itemize}
% subsection Temporização da Mensagem (end)

\subsection{Visão geral dos protocolos} % (fold)
\label{sub:Visão geral dos protocolos}
Os protocolos de rede definem um formato comum e um conjunto de regras para a troca 
de mensagens entre dispositivos. Os protocolos são implementados por dispositivos 
finais e dispositivos intermediários em software, hardware e ambos.

Os protocolos são diversos, e cada um deles atende a uma necessidade específica, sendo 
eles os seguintes para cada local de atuação do mesmo: 

\begin{itemize}
  \item \textbf{Protocolos de comunicação em rede: } Permitem que dois ou mais dispositivos se comuniquem através de um ou mais redes, são usados os IP, TCP e HTTP. 
  \item \textbf{Protocolos de segurança de rede: }Protegem os dados para fornecer autenticação, integridade dos dados e criptografia, alguns deles são: SSH, TLS e SSL. 
  \item \textbf{Protocolos de roteamento: } permitem que os roteadores troquem informações de rota, compare caminhos e selecionar a melhor rota até o destino, alguns exemplos são o OSPF e BGP. 
  \item \textbf{Protocolos de descoberta de serviços: } São usados para a detecção automática de dispositivos ou serviços, os mais famosos são DHCP e DNS.
\end{itemize}
% subsection Visão geral dos protocolos (end)


\section{Protocolo de rede} % (fold)
\label{sec:Protocolo de rede}
Os protocolos de comunicação são os responsáveis por uma variedade de funções 
necessárias para a comunicação entre redes de dispositivos finais, temos o seguinte: 

\begin{center}
  \begin{tabular}{ | m{3cm} | m{7cm} |}
\hline
\textbf{Função} & \textbf{Descrição} \\
\hline
\textbf{Endereçamento} & Identifica  o remetente e o destinatário da mensagem, usando o endereço definido no esquema, que pode ser definido com protocolo ethernet, IPv4 ou IPv6.\\
\hline 
\textbf{Sequenciamento} & Esta função rotula exclusivamente cada segmento de dados transmitidos, usando as informações de sequenciamento para remontar a informação corretamente.\\
\hline
\textbf{Deetecção de Erros } & Esta função é usada para determinar se os dados foram corrompidos durante a transmissão. Vários protocolos que fornecem detecção de erros incluem Ethernet, IPv4, IPV6 e TCP.\\
\hline
\textbf{Interface de aplicação} & Esta função contém informações usadas para processo a processo comunicações entre aplicações de rede.\\
\hline
\end{tabular}
\end{center}

\subsection{Interação de protocolos} % (fold)
\label{sub:Interação de protocolos}
Uma mensagem enviada através de uma rede de computadores normalmente requer o uso de 
vários protocolos, cada um com suas próprias funções e formato. E cada um possui 
a seguinte especificidade:

\begin{itemize}
  \item \textbf{Protocolo de Transferência de Hipertexto(HTTP): }Este protocolo controla a maneira como um servidor web e um cliente web interagem. Ele define o conteúdo e formato das solicitações e respostas trocadas entre o cliente e o servidor.  
  \item \textbf{Transmission Control Protocol(TCP): } Este protocolo gerencia as conversas individuais, ele garante que a entrega é confiável dsa informações e gerenciar o controle de fluxo entre os dispositivos finais. 
  \item \textbf{Protocolo Internet(IP): } Este protocolo é responsável por entregar mensagens do remetente para o receptor. IP e é usado por roteadores para encaminhar como mensagens em várias redes.
  \item \textbf{Ethernet: }Entrega as mensagens de uma NIC para outra, na mesme rede local LAN.
\end{itemize}
% subsection Interação de protocolos (end)

\subsection{Conjunto de protocolos de rede} % (fold)
\label{sub:Conjunto de protocolos de rede}
Os protocolos devem ser capazes de trabalhar com outros protocolos para que sua 
experiência on-line lhe dê tudo oque precisa para comunicações de rede. Os conjuntos 
de protocolo são projetados para trabalhar entre si sem problemas. Um conjunto de 
protocolos é um grupo de protocolos inter-relacionados necessários para executar uma 
função de comunicação. 

Uma das melhores formas de entender como os protocolos interagem entre si, é o 
entendimento de que eles funcionam como se fosse uma pilha. Uma pilha de protocolos 
mostra como os protocolos individuais são implementados.
% subsection Conjunto de protocolos de rede(end)

\subsection{Evolução do conjunto de protocolos} % (fold)
\label{sub:Evolução do conjunto de protocolos}
Uma suíte de protocolos é um grupo de protocolos que funciona em conjunto para fornecer
serviços abrangentes de comunicação em redes. Os protocolos evoluíram e diversos 
padrões foram desenvolvidos. 

\begin{itemize}
  \item \textbf{Internet Protocol Suite ou TCP/IP: } Este é o conjunto de protocolos mais comuns e relevantes usados nos dias de hoje, é um protocolo padrão aberto. 
  \item \textbf{Protocolos de Interconexão de Sistemas Abertos(OSI): }É um conjunto de protocolos antigo que incluia um modelo de sete camadas. O modelo de referência categoriza as funções de seus protocolos. O OSI é conhecido principalmente por modelos em camadas, e ele foi amplamente substituído pelo TCP/IP. 
  \item \textbf{AppleTalk: }Conjunto de protocolos proprietário da Apple
  \item \textbf{Novell NetWare: }Conjunto de protocolos proprietários de curta direção e sistema operacional de rede desenvolvido em 1983.
\end{itemize}
% subsection Evolução do conjunto de protocolos (end)

\subsection{Suíte de protocolos tcp/ip} % (fold)
\label{sub:Suíte de protocolos tcp/ip}
A suíte TCP/IP inclui dentro dele outros protocolos que 
ajudam no desenvolvimento e funcionamento dessa suíte,
e muitos deles se localizam nas camadas de internet e 
alguns se localizam na camada de aplicação.

\begin{itemize}
  \item \textbf{Conjunto de protocolos de padrão aberto: }Está disponível gratuitamente ao público e pode ser usado por qualquer fornecedor em seu hardware ou software. 

  \item \textbf{Conjunto de protocolos com base em padrões: }Isso significa que foi endossado pela indústria de rede e aprovado por uma organização de padrões. Isso garante que produtos diferentes interoperem.
\end{itemize}

Os protocolos possuem sua denominação e área de atuação, pois eles atuam em 
modos específicos dentro da comunicação.

% subsection Suíte de protocolos tcp/ip (end)

\subsection{Processo de Comunicação} % (fold)
\label{sub:Processo de Comunicação}

O processo de comunicação dentro de servidores, é feito por meio de quadros ethernet. 
esses quadros funcionam encapsulando dados do protocolo superior ou inferior, e 
processamendo eles para que o quadro seguinte refaça o encapsulamento ou envio os dados. 
Temos que esse processo é que os dados, são implementados dentro de um segmento TCP,
e depois reencapsulado dentro de um pacote IP, que ao ter adicionado a si um 
cabeçalho HTTP, pode em vias de fato virar um quadro ethernet. 
% subsection Processo de Comunicação (end)


\subsection{Modelos de Referência} % (fold)
\label{sub:Modelos de Referência}
O uso de um modelo padronizado em camadas, auxilia no entendimento de como a 
comunicação em rede funciona. e o modelo em camadas ajuda a descrever protocolos e 
operações de redes, que nos auxiliam no projeto de protocolos pois operam em uma 
camada específica determinada. Fomenta a concorrência de produtos de diferentes 
fornecedores e impede alteração de tecnologias ou capacidade. 

\subsubsection{Modelo referência OSI} % (fold)
\label{sec:Modelo referência OSI}
O modelo OSI fornece uma extensa lista de funções e serviços que ocorrem em suas 
camadas, e pelo fato dele ser mais modularizado que o modelo TCP/IP, eke possui mais 
protocolos específicos. Esses protocolos descrevem oque deve ser feito, mas não a forma
que deve ser, também descrevendo a forma de interação de cada camada com as camadas 
vizinhas.

\newpage

\begin{center}
  \begin{tabular}{| m{3cm} | m{7cm} |}
  \hline
  \textbf{Camada do Modelo OSI} & \textbf{Descrição} \\
  \hline
  7 - Aplicação & Contém os protocolos usados para processo a processo comunicativo. \\
  \hline
  6 - Apresentação & Fornece uma representação comum dos dados transferidos entre serviços da camada de aplicativo. \\ 
  \hline
  5 - Sessão & A camada de sessão fornece serviços para camada de apresentação para organizar o diálogo e gerenciar intercâmbio de dados. \\
  \hline
  4 - Transporte & A camada de transporte define serviços para segmentar, transferir e remontar dados para comunicações individuais entre os dispositivos finais. \\
  \hline
  3 - Rede & Fornece serviços para troca de partes individuais de dados dos quadros entre dispositivos finais. \\
  \hline
  2 - Enlace de Dados & Descrevem métodos para troca de dados entre dispositivos comuns.\\
  \hline  
  1 - Físico & Descrevem as partes físicas e processuaus para manter e desativar as conexões físicas para transmissão de bits \\

 
\hline
\end{tabular}
\end{center}
% subsubsection Modelo referência OSI (end)

\subsubsection{Modelo de referência TCP/IP} % (fold)
\label{sec:Modelo de referência TCP/IP}
Ele diferente do modelo OSI, possui menos divisões, sendo somente 4 grade divisões. Ele
foi craido nos anos 70 e corresponde à estrutura de um conjunto específico de protocolos
 porque descreve as funções que ocorrem em cada camada por meio de protocolos delas. 

\begin{center}
\begin{tabular}{| m{3cm} | m{7cm} |}

  \hline
  \textbf{Camda do modelo TCP/IP} & \textbf{Descrição} \\
  \hline  
  4 - Aplicação & Representa dados para o usuário, além do controle de codificação e do diálogo \\
  \hline
  3 - Transporte & Permite a comunicação entre vparios dispositivos diferentes em redes distintas.\\
  \hline
  2 - Internet & Determina o melhor caminho pela rede\\
  \hline
  1 - Acesso à rede & Controla os dispositivos de hardware e os meios físicos que formam a rede.\\

  \hline
\end{tabular}
\end{center}
% subsubsection Modelo de referência TCP/IP (end)
% subsection Modelos de Referência (end)

\subsection{Encapsulamento de Dados} % (fold)
\label{sub:Encapsulamento de Dados}
\subsubsection{Segmentando Mensagens} % (fold)
\label{sec:Segmentando Mensagens}
Em teoria, uma únia comunicação poderia ser enviada através de uma rede por meio de um 
fluxo maciço de bits. No entanto, isso criaria problemas comunicativos entre os 
dispositivos conectados nessa rede que precisassem usar os mesmo canais comunicativos, 
resultando em atrasos consideráveis de comunicação. Por isso, as mensagens são 
dividias em pequenos pacotes que são mais fáceis de serem gerenciáveis pela rede. 
Segmentação é o processo de dividir um fluxo de dados em unidades menores pra 
transmissão. Ela é necessária porque as redes de dados usam o conjunto de protocolos 
TCP/IP para enviar dados em pacotes IP individuais. Isso promove duas coisas, 
\textbf{aumento de velocidade de transmissão de dados} e \textbf{aumento de eficiência}.

Esses dois elementos permitem a \textbf{multiplexação}, que é quando duas conversas 
diferentes são intercaladas e gerenciadas na rede ao mesmo tempo.
% subsubsection Segmentando Mensagens (end)

\subsubsection{Sequenciamento} % (fold)
\label{sec:Sequenciamento}
Na comunicação em rede, cada segmento de mensagem deve passar por um processo 
semelhante para garantir que chege ao destino correto e possa ser remontado no 
conteúdo da mensagem original. Para isso, temos que o sequenciamento de pacotes
é feito para que a ordem de entrega seja feita de maneira correta, e caso algum 
pacote seja perdido, pode ser identificado e recuperado.
% subsubsection Sequenciamento (end)
% subsection Encapsulamento de Dados (end)


\subsection{Acesso a dados} % (fold)
\label{sub: Acesso a dados}
As camada de rede e de enlace de ados são responsáveis por entregar os dados do 
dispositivo origem para o dispositivo de destino. Ois protocolos nas duas camadas 
contêm um endereço de origem e de destino, mas seus endereços têm finalidades 
diferentes:

\begin{itemize}
  \item \textbf{Endereços de origem e destino da camada de rede: } Responsável por entregar o pacote IP da origem original ao destino final.
  \item \textbf{Endereços de origem e destino na camada de enlace de dados: } Responsável por fornecer o quadro de enlace de dados de uma palca de interface de rede para outra na mesma rede.
\end{itemize}

\subsubsection{Endereço Lógico da Camada 3} % (fold)
\label{sec:Endereço Lógico da Camada 3}
Um endereço IP é o endereço lógico da camada de rede, usando para entregar o pacote IP 
da origem original ao destino final. O pacote Ip contém o endereço IP de origem e de 
destino do pacote. Ele também contém dentro de si, um endereço IP com duas partes, sendo elas: 

\begin{itemize}
  \item \textbf{Parte da rede (IPv4) ou Prefixo(IPv6): }A parte mais à esquerda do endereço que indica qual rede o endereço IP é membro, Todos os dispositivos na mesma rede terão a mesma parte da rede.
  \item \textbf{Parte do host (IPv4) ou ID da interface (IPv6): } A parte restante do endereço que identifica um dispositivo específico na rede. É uma parte exclusiva para cada dispositivo ou na interface na rede.
\end{itemize}
% subsubsection Endereço Lógico da Camada 3 (end)

\subsubsection{Dispositivos na mesma rede} % (fold)
\label{sec:Dispositivos na mesma rede}
Quando temos dois dispositivos computacionais dentro de uma mesma rede, sua 
comunicação tem como ideia, que parte do endereçamento da rede IPv4 de origem e 
destino, possuem o mesmo endereçamento, como mostrado nos seguintes dados: 

\begin{itemize}
  \item \textbf{Endereço IPv4 origem: } Endereço IPv4 de origem é 192.168.1.110
  \item \textbf{Endereço IPv4 destino: }O endereço IPv4 de recebimento do FTP é 192.168.1.9
\end{itemize}

Temos que na camada de rede e dentro do cabeçalho do endereçamento IP, temos as 
informações que definem que estes dois dispositivos estão presentes dentro da 
mesma rede. 

\subsubsection{Função de endereço na camada de enlace na mesma rede} % (fold)
\label{sec:Função de endereço na camada de enlace na mesma rede}
Quando a origem e o destino do pacote IP estiverem na mesma rede, o quadro de enlace 
de dados será enviado diretamente para o dispositivo receptor. Em uma rede Ethernet, 
os endereços do link de dados são conhecidos como endereços MAC. Esses endereços são 
embutidos fisicamente nas NIC Ethernet. 
% subsubsection Função de endereço na camada de enlace na mesma rede (end)


\subsubsection{Dispositivos em rede remota} % (fold)
\label{sec:Dispositivos em rede remota}
Quando tivemos elementos comunicativos em redes distintas, temos que os enedereço 
IP de origem e de destinos representação hosts em redes diferentes. Isso é indicado 
pela porção da rede dentro do endereço IP. 

\begin{itemize}
  \item \textbf{Endereço IPv4 origem: } Endereço do computador: 192.168.1.110
  \item \textbf{Endereço IPv4 destino: } Endereço servidor web: 172.16.1.99
\end{itemize}

Para que haja a comunicação, temos o uso da camada de enlace de dados, que a partir 
dela podemos por meio de dispositivos intermediários, acessar o host de destino,
assim, os dados e o quadro Ethernet devem ser enviados primeiro ao roteador, que também 
é conhecdo como \textbf{gateway padrão}. 

Com os endereços de enlace de dados, temos que a camada 2 tem a finalidade de fornecer 
o quadro de enlace de dados de uma interface de rede para outra na mesma rede.

% subsubsection Dispositivos em rede remota (end)
% subsubsection Dispositivos na mesma rede (end)
% subsection Acesso a dados (end)

\section{Camada Física} % (fold)
\label{sec:Camada Física}
\subsection{Características da camada física} % (fold)
\label{sub:Características da camada física}
Os padrões de comunicação da camda física abordam três áreas funcionais, que são: 

\begin{itemize}
  \item Componentes físicos
  \item Codificação 
  \item Sinalização
\end{itemize}

Os \textbf{componentes físicos} são os dispositivos de hardware eletrônico e conectores 
que transmitem os sinais que representam os bits. Esses componentes como NICs, interfaces e conectores são especificados nos padrões associados à camada física.

Por outro lado, a \textbf{codificação} é um método para converter um fluxo de bits 
de dados em um código pré-definido. Os códigos são agrupamentos de bits usados para 
fornecer um padrão previsível que pode ser reconhecido tanto pelo emissor quanto pelo 
recemtpr. A codificação é o método ou o padrão usado para representar as informações 
digitais. 

Por fim, temos que a \textbf{sinalização} é a forma como a camada física gera sinais 
elétricos ou ópticos, e até mesmo sem fio, que representam os valores de "1" e "0" no 
meio físico. A maneira como os bits são representados é chamado de método de 
sinalização. Os padrões da camada física devem definir que tipo de sinal representa 
o pulso ligado e o pulso desligado.

\subsubsection{Largura da Banda} % (fold)
\label{sec:Largura da Banda}
Meios físicos diferentes aceitam a transferência de bits a taxas diferentes. A 
transferência de dados é geralmente discuta em termos de largura de banda, que é a capacidade no qual um meio pode transportar dados. A largura digital mede a quantidade de dados que podem fluir em um lugar para outro durante determiando tempo, esse valor normalmente é medido em kilobits, ou megabits por segundo.
% subsubsection Largura da Banda (end)

\subsubsection{Terminologia} % (fold)
\label{sec:Terminologia}
A \textbf{latência} se refere ao tempo necessário para que os dados viajem de um ponto 
ao outro, isso incluindo seus atrasos. Em uma internet com vários segmentos. A taxa 
de transferência não pode ser mais rápida que o link mais lento no cainho de origem 
ao caminho de destino. Mesmo que todos ou a maioria dos segmentos tenham alta largura 
de banda, será necessário apenas um segmento no caminho com baixa taxa de tranferência 
para ciar um gargalo na taxa de trasferência total da rede. 

\textbf{Taxa de transferência} é o nome dado a medida de bits que atravesa uma mídia 
durante um período. A taxa de trnasferência normalmente é afetada, e costuma ser menor 
que a máxima configura dentro da camada física devido a fatores como a quantidade de 
tráfego, o tipo do mesmo, e a latência craida pelo número de dispositivos de rede 
que são encontrados entre a origem e o destino. 
% subsubsection Terminologia (end)

\subsubsection{Características dos cabeamento de cobre} % (fold)
\label{sec:Características dos cabeamento de cobre}
O cabeamento de cobre é o tipo de conexão mais comum que existem no cabeamento de redes 
hoje em dia pelo fato de ser barato e fácil de ser instalado. Um dos problema atrelado 
ao seu uso, é o fato de ter uma distânci limitada pela interferência de sinal. 
Os dados são repsentados por meio de pulsos elétricos. E com isso, temos que devido 
à essa especificidade técnica, os dois tipos de interferências mais comuns de 
ocorrerem dentro do usso dos cabos de cobre, são 

\begin{itemize}
  \item \textbf{Interferência eletromagnética (EMI) radiofrequência(RFI): } Podem distorcer ou corromper os sinais de dados que estão sendo transportados pela mídia de cobre. Possíveis fontes dessas itnerferências são dispositivos de ondas de rádio e eletromagnéticos.
  \item \textbf{Diafonia: }É a pertubação causada pelos campos elétricos ou magnéticos de um sinal em um fio para o sinal em um fio adjacente. Nos circuitos de telefone, a diafoni pode fazer com que parte de outra conversa de voz de um circuito adjacente seja ouvida. Especificamente quando uma corrente elétrica flui através do cabo.
\end{itemize}

Alguns cabos possuem especificidades técnicas na sua construção que impede que esses 
tipos de interferências possam ocorrer, como por exemplo, os 
\textbf{cabos de par trançado (UTP)}, \textbf{cabo de par trançado blindado(STP)} e 
\textbf{Cabo coaxial}.
% subsubsection Características dos cabeamento de cobre (end)

\subsubsection{Cabeamento UTP} % (fold)
\label{sec:Cabeamento UTP}
Quando usamos o UTP como meio de rede , o cabeamento 
consiste em quatro pares de fios entrelaçados com cores 
específicas de cada par, que são envoltos em um tubo de 
plástico para proteção. 

Pelo fato desse tipo de fio não usar blindagem em sua 
construção, os projetistas encontraram outra maneira de 
limitar a diafonia, e isso pode ser feito por meio de 
duas técnicas: 

\begin{itemize}
  \item \textbf{Cancelamento: }Emparelham os fios em um circuto. Quando dois fios de um circutos são colocados próximos um ao outro, seus campos magnéticos serão opostos.
  \item \textbf{Variando o número de torções: }Aumenta o efeit ode cancelamento de fios em circutos emparelhados, mas deve seguir especificações de quantas tranças são permitidas por metro.
\end{itemize}

% subsubsection Cabeamento UTP (end)

\subsubsection{Padrões e Conectores de cabeamento UTP} % (fold)
\label{sec:Padrões e Conectores de cabeamento UTP}
O cabeamento possui alguns dos seus elementos definidos 
por padrões para facilitar a forma como o manejamento e o 
trabalho com eles é feito, pois padroniza os mecanismos e 
a forma como se trabalha com eles. Alguns desses elementos 
padronizados são: 

\begin{itemize}
  \item Tipos de cabos 
  \item Comprimento do cabo 
  \item Conectores 
  \item Terminação do Cabo 
  \item Métodos de teste do cabo
\end{itemize}

Existem divisões de categorias entre os cabos, cada uma 
delas corresponde a quantidade de dados que podem ser 
transferidos no cabo, sendo eles respectivamente: 

\begin{itemize}
  \item Categoria 5 -> 100Mbps e 1000Mbps
  \item Categoria 6 -> 10Gbps
  \item Categoria 7 -> 10Gbps 
  \item Categoria 8 -> 40 Gbps 
\end{itemize}

O cabo por possuir essas esécificidades, possui também a 
especificação da sua conecxão física, a mesma deve ser 
feita por meio de conectores RJ-45, tanto os plugs como 
os sockets do cabo.

% subsubsection Padrões e Conectores de cabeamento UTP (end)

\subsubsection{Cabeamento de fibra ótica} % (fold)
\label{sec:Cabeamento de fibra ótica}
O cabeamento de fibra ótica, consiste em uma fibra que 
normalmente é feita de vidro, e que pode ser de dois tipos, 
a de monomodo e a de multimodo.

A fibra de \textbf{monomodo} consiste em um núcleo em que 
um único fio de lux é enviado, e ele é mais usado para 
longas distâncias. Por outro lado, a fibra de 
\textbf{multimodo} possui um núcleo mais dilatado, e é 
mais usado em curtas distâncias, por outro lado, diversos 
feixes de luzes são enviados ao mesmo tempo.
% subsubsection Cabeamento de fibra ótica (end)
% subsection Características da camada física (end)
% section Camada Física (end)
\newpage
\section{Camada de Enlace de Dados} % (fold)
\label{sec:Camada de Enlace de Dados}
A camada de enlace de dados prepara os dados da rede para 
a rede física, além de também ser responsável pela placa de 
interface da rede (NIC).
Na camada de enlace, é que está presente as subcamadas de 
\textbf{LLC} e de \textbf{MAC}. A camada MAC é de suma 
importância pelo fato dela delimitar os quadros, definir 
os endereçamentos, e detectar possíveis erros de transmissão

\subsection{Topologias} % (fold)
\label{sub:Topologias}
A topologia de um rede é a organização ou relacionamento, 
dos dispositivos de rede e as interconexões entre eles. 
Existem dois tipos de topologias usadas para descrever 
as redes: 

A topologia \textbf{física} identifica as conexões físicas 
e a forma como os dispositivos finais e intermediários são 
interconectados. A topologia também pode incluir a 
localização específica do dispositivo, como o número do rack 
e o número do equipamento. Por outro lado, a topologia 
\textbf{lógica} refere-se à maneira como uma rede transfere
quandros de um nó para o próximo. Eta topologia identifica 
conexões virtuais usando interfaces de dispositivos e 
esquemas de endereçamento IP. As principais topologias 
podem ser dividas entre as topologias focadas em WAN e 
outras para LAN, sendo elas: 

\begin{itemize}
  \item WAN
  \begin{itemize}
    \item Ponto a Ponto
    \item Estrela 
    \item Malha
  \end{itemize}
  \item LAN 
  \begin{itemize}
    \item Barramento
    \item Anel
  \end{itemize}
\end{itemize}

\subsubsection{Formas de comunicação} % (fold)
\label{sec:Formas de comunicação}
A comunicação pode ser feita de duas forma, que são chamadas 
half-duplex ou full-duplex. A comunicação \textbf{half-duplex} é feita por ambos dispositivos, e pode transmitir e 
receber dados, mas isso não pode ser feito de forma 
simultânea. Ele permite que apenas um dispositivo envia ou 
receba por vez na mídia compartilhada. Por outro lado, 
a comunicação \textbf{full-duplex} permite a transmissão e 
o envio simultâneo de dados. 

\subsubsection{Métodos de controle de acesso} % (fold)
\label{sec:Métodos de controle de acesso}
São exemplos de redes de multiacesso, as redes LAN e WAN. 
Esse tipo de rede pode ter dois ou mais dispositivos finais 
tentando acessar a rede simultaneamente. Existem dois  
métodos básicos de controle de acesso para para meio físico 
compartilhado. 

\begin{itemize}
  \item Acesso baseado em contenção 
  \item Acesso controlado.
\end{itemize}  
% subsubsection Métodos de controle de acesso (end)
Em uma rede de multiacesso controlada, cada nó tem seu 
próprio tempo para usar o meio. Eses tipos determinísticos 
de redes herdadas são ineficientes porque um dispositivo 
deve aguardar sua vez para acessar o meio. Exemplos de 
redes multiacesso que usam esses controles são os de 
aneis de token legados e ARCNET.

\subsubsection{Acesso baseado em contenção - CSMA} % (fold)
\label{sec:Acesso baseado em contenção - CSMA}
Tipos de redes que possuem esse tipo de acesso, são 
redes como Lan sem fio, Lan ethernet na topologia de 
barramento legado e LANs Etherhet herdadas usando hub. 

Essas redes costumam operar no modo half-duplex, oque 
significa que apenas um dispositivo pode enviar ou receber
de cada vez. 
% subsubsection Acesso baseado em contenção - CSMA (end)
% subsection Topologias (end)

\subsection{Quadro de enlace de dados} % (fold)
\label{sub:Quadro de enlace de dados}
A camada de link de dados prepara os dados encapsulados para
o transporte pela mídia local, encapsulando-o com um 
cabeçalho e um trailer para criar um quadro. O protocolo de 
link de dados é responsável pelas comunicações de NIC para 
NIC dentro da mesma rede. Embora existam muitos protocolos
de camada de enalce de dados diferentes que descrevem os 
quadros de caamda de enlace de dados. Cada quadro costuma 
ter: 

\begin{itemize}
  \item Cabeçalho
  \item Dados
  \item Trailer
\end{itemize}

Ao contrário de outros protocolos de encapsulamento, a 
camada de link de dados acrescenta informações na forma 
de um trailer no final do quadro. Todos os protocolos 
da camada de enlace de dados encapsulam os dados dentro do
campo de dados do quadro. A estrutura do quadro e os 
os campos contidos no cabeçalho e trailer variam de 
acordo com o protocolo

Não há
% subsection Qadro de enlace de dados(end)



% section Camada de Enlace de Dados (end)
\end{document}
