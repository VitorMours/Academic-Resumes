\documentclass[12pt a4paper]{paper}
\usepackage{array}

\begin{document} 
\section{Protocolos e Modelos} % (fold)
\label{sec:Protocolos e Modelos}
\subsection{Fundamentos da Comunicação} % (fold)
\label{sub:Fundamentos da Comunicação}
As redes variam em tamanho, forma e função. Elas podem ser tão complexas quanto os 
dispositivos conectados, ou tão simples quanto dois computadores conectados diretamente
um ao outro com um único cabo e qualquer outra coisa.
As pessoas trocam ideias usando vários métodos de comunicação diferentes. No entanto, 
todos os métodos de comunicação têm os seguintes três elementos em comum:

\begin{itemize}
  \item \textbf{Fonte da Mensagem: } As fontes da mensagem são pessoas ou dispositivos eletrônicos que precisam enviar uma mensagem para outro lugar.
  \item \textbf{Destino da Mensagem: }O destino recebe a mensaem e a interpreta.
  \item \textbf{Canal: }Consiste na mídia que fornece o caminho pelo qual a mensagem viaja da oridem ao destino.
\end{itemize}
% subsection Fundamentos da Comunicação (end)

\subsection{Protocolos de Comunicação} % (fold)
\label{sub:Protocolos de Comunicação}
O envio da mensagem, é regido por regras que são chamadas de protocolos. 
Esses protocolos são específicos ao tipo de método de comunicação que estão sendo usado.
Em nossa comunicação pessoal do dia-a-dia, as regras que usamos para nos comunicar 
em uma mídia.
% subsection Protocolos de Comunicação (end)

\subsection{Requisitos de protocolo de rede} % (fold)
\label{sub:Requisitos de protocolo de rede}
Os protocolos usado nas comunicações de rede compartilham muitas dessas características
fundamentais. Além de identificar a origem e o destino, os protocolos de computadores e 
de redes definem os detalhes sobre como uma mensagem é trnasmitida por uma rede. 
Eles possuem os seguintes requisitos:

\begin{itemize}
  \item Codificação de mensagens.
  \item Formatação e encapsulamento de mensagens. 
  \item Tamanho da mensagem. 
  \item Tempo da mensagem. 
  \item Opção de envio de mensagem.
\end{itemize}
% subsection Requisitos de protocolo de rede (end)

\subsection{Codificação de Mensagens} % (fold)
\label{sub:Codificação de Mensagens}
Uma das primeiras etapas para enviar uma mensagem, é codificar ela. A codificação é o 
processo de conversão de informações em outra forma aceitável para a transmissão. A 
decodificação reverte esse processo para interpretar as informações.
% subsection Codificação de Mensagens (end)

\subsection{Formatação e Encapsulamento de Mensagens} % (fold)
\label{sub:Formatação e Encapsulamento de Mensagens}
Quando uma mensagem é enviada da origem para o destino, deve usar um formato e 
estrutura específica, para que seu uso seja o mais eficiente possível. Dentro da rede, 
isso se transforma com o uso de IP, como uma função de localização para identificar a 
origem e o destino da mensagem.
% subsection Formatação e Encapsulamento de Mensagens (end)

\subsection{Tamanho da Mensagem} % (fold)
\label{sub:Tamanho da Mensagem}
Quando uma mensagem é muita longa, para ela ser enviada de um host para o outro, é 
necessário dividir a mensagem em partes menores, pois isso facilita o transporte e a 
detecção de possíveis falhas dentro da nossa comunicação




% subsection Tamanho da Mensagem (end)

\subsection{Temporização da Mensagem} % (fold)
\label{sub:Temporização da Mensagem}
O tempo que as mensagens possuem de vida também é extremamente importante, e isso é 
controlado por meio de três elementos que estão presente, sendo eles: 

\begin{itemize}
  \item \textbf{Controle de Fluxo: } É o processo de gerenciamento  da taxa de transmissão dos dados. o controle define quanta informação pode ser enviada e a velocidade com que pode ser entrege.
  \item \textbf{Tempo limite de resposta: } Se uma pessoa fizer uma pergunta e nçao ouvir uma resposta dentro de um período aceitável de tempo, entende que a resposta nunca deve chegar, o mesmo funciona para os computadores. Os hosts da rede usam protocolos de rede e especificam quanto tempo de espera pode ser ofertado para que uma ação seja executada.
  \item \textbf{Método de Acesso: } Ajuda a determinar quando alguém pode enviar uma mensagem. Quando um dispositivo deseja transmitir uma mensagem, sendo por fio ou sem fio, deve existir um tipo de conecção, podendo ser ela por LAN ou WLAN, respectivamente.
\end{itemize}
% subsection Temporização da Mensagem (end)

\subsection{Visão geral dos protocolos} % (fold)
\label{sub:Visão geral dos protocolos}
Os protocolos de rede definem um formato comum e um conjunto de regras para a troca 
de mensagens entre dispositivos. Os protocolos são implementados por dispositivos 
finais e dispositivos intermediários em software, hardware e ambos.

Os protocolos são diversos, e cada um deles atende a uma necessidade específica, sendo 
eles os seguintes para cada local de atuação do mesmo: 

\begin{itemize}
  \item \textbf{Protocolos de comunicação em rede: } Permitem que dois ou mais dispositivos se comuniquem através de um ou mais redes, são usados os IP, TCP e HTTP. 
  \item \textbf{Protocolos de segurança de rede: }Protegem os dados para fornecer autenticação, integridade dos dados e criptografia, alguns deles são: SSH, TLS e SSL. 
  \item \textbf{Protocolos de roteamento: } permitem que os roteadores troquem informações de rota, compare caminhos e selecionar a melhor rota até o destino, alguns exemplos são o OSPF e BGP. 
  \item \textbf{Protocolos de descoberta de serviços: } São usados para a detecção automática de dispositivos ou serviços, os mais famosos são DHCP e DNS.
\end{itemize}
% subsection Visão geral dos protocolos (end)


\section{Protocolo de rede} % (fold)
\label{sec:Protocolo de rede}
Os protocolos de comunicação são os responsáveis por uma variedade de funções 
necessárias para a comunicação entre redes de dispositivos finais, temos o seguinte: 

\begin{center}
  \begin{tabular}{ | m{3cm} | m{7cm} |}
\hline
\textbf{Função} & \textbf{Descrição} \\
\hline
\textbf{Endereçamento} & Identifica  o remetente e o destinatário da mensagem, usando o endereço definido no esquema, que pode ser definido com protocolo ethernet, IPv4 ou IPv6.\\
\hline 
\textbf{Sequenciamento} & Esta função rotula exclusivamente cada segmento de dados transmitidos, usando as informações de sequenciamento para remontar a informação corretamente.\\
\hline
\textbf{Deetecção de Erros } & Esta função é usada para determinar se os dados foram corrompidos durante a transmissão. Vários protocolos que fornecem detecção de erros incluem Ethernet, IPv4, IPV6 e TCP.\\
\hline
\textbf{Interface de aplicação} & Esta função contém informações usadas para processo a processo comunicações entre aplicações de rede.\\
\hline
\end{tabular}
\end{center}

\subsection{Interação de protocolos} % (fold)
\label{sub:Interação de protocolos}
Uma mensagem enviada através de uma rede de computadores normalmente requer o uso de 
vários protocolos, cada um com suas próprias funções e formato. E cada um possui 
a seguinte especificidade:

\begin{itemize}
  \item \textbf{Protocolo de Transferência de Hipertexto(HTTP): }Este protocolo controla a maneira como um servidor web e um cliente web interagem. Ele define o conteúdo e formato das solicitações e respostas trocadas entre o cliente e o servidor.  
  \item \textbf{Transmission Control Protocol(TCP): } Este protocolo gerencia as conversas individuais, ele garante que a entrega é confiável dsa informações e gerenciar o controle de fluxo entre os dispositivos finais. 
  \item \textbf{Protocolo Internet(IP): } Este protocolo é responsável por entregar mensagens do remetente para o receptor. IP e é usado por roteadores para encaminhar como mensagens em várias redes.
  \item \textbf{Ethernet: }Entrega as mensagens de uma NIC para outra, na mesme rede local LAN.
\end{itemize}
% subsection Interação de protocolos (end)

\subsection{Conjunto de protocolos de rede} % (fold)
\label{sub:Conjunto de protocolos de rede}
Os protocolos devem ser capazes de trabalhar com outros protocolos para que sua 
experiência on-line lhe dê tudo oque precisa para comunicações de rede. Os conjuntos 
de protocolo são projetados para trabalhar entre si sem problemas. Um conjunto de 
protocolos é um grupo de protocolos inter-relacionados necessários para executar uma 
função de comunicação. 

Uma das melhores formas de entender como os protocolos interagem entre si, é o 
entendimento de que eles funcionam como se fosse uma pilha. Uma pilha de protocolos 
mostra como os protocolos individuais são implementados.
% subsection Conjunto de protocolos de rede(end)

\subsection{Evolução do conjunto de protocolos} % (fold)
\label{sub:Evolução do conjunto de protocolos}
Uma suíte de protocolos é um grupo de protocolos que funciona em conjunto para fornecer
serviços abrangentes de comunicação em redes. Os protocolos evoluíram e diversos 
padrões foram desenvolvidos. 

\begin{itemize}
  \item \textbf{Internet Protocol Suite ou TCP/IP: } Este é o conjunto de protocolos mais comuns e relevantes usados nos dias de hoje, é um protocolo padrão aberto. 
  \item \textbf{Protocolos de Interconexão de Sistemas Abertos(OSI): }É um conjunto de protocolos antigo que incluia um modelo de sete camadas. O modelo de referência categoriza as funções de seus protocolos. O OSI é conhecido principalmente por modelos em camadas, e ele foi amplamente substituído pelo TCP/IP. 
  \item \textbf{AppleTalk: }Conjunto de protocolos proprietário da Apple
  \item \textbf{Novell NetWare: }Conjunto de protocolos proprietários de curta direção e sistema operacional de rede desenvolvido em 1983.
\end{itemize}
% subsection Evolução do conjunto de protocolos (end)

\subsection{Suíte de protocolos tcp/ip} % (fold)
\label{sub:Suíte de protocolos tcp/ip}
A suíte TCP/IP inclui dentro dele outros protocolos que 
ajudam no desenvolvimento e funcionamento dessa suíte,
e muitos deles se localizam nas camadas de internet e 
alguns se localizam na camada de aplicação.

\begin{itemize}
  \item \textbf{Conjunto de protocolos de padrão aberto: }Está disponível gratuitamente ao público e pode ser usado por qualquer fornecedor em seu hardware ou software. 

  \item \textbf{Conjunto de protocolos com base em padrões: }Isso significa que foi endossado pela indústria de rede e aprovado por uma organização de padrões. Isso garante que produtos diferentes interoperem.
\end{itemize}

Os protocolos possuem sua denominação e área de atuação, pois eles atuam em 
modos específicos dentro da comunicação.

% subsection Suíte de protocolos tcp/ip (end)

\subsection{Processo de Comunicação} % (fold)
\label{sub:Processo de Comunicação}

O processo de comunicação dentro de servidores, é feito por meio de quadros ethernet. 
esses quadros funcionam encapsulando dados do protocolo superior ou inferior, e 
processamendo eles para que o quadro seguinte refaça o encapsulamento ou envio os dados. 
Temos que esse processo é que os dados, são implementados dentro de um segmento TCP,
e depois reencapsulado dentro de um pacote IP, que ao ter adicionado a si um 
cabeçalho HTTP, pode em vias de fato virar um quadro ethernet. 
% subsection Processo de Comunicação (end)


\subsection{Modelos de Referência} % (fold)
\label{sub:Modelos de Referência}
O uso de um modelo padronizado em camadas, auxilia no entendimento de como a 
comunicação em rede funciona. e o modelo em camadas ajuda a descrever protocolos e 
operações de redes, que nos auxiliam no projeto de protocolos pois operam em uma 
camada específica determinada. Fomenta a concorrência de produtos de diferentes 
fornecedores e impede alteração de tecnologias ou capacidade. 

\subsubsection{Modelo referência OSI} % (fold)
\label{sec:Modelo referência OSI}
O modelo OSI fornece uma extensa lista de funções e serviços que ocorrem em suas 
camadas, e pelo fato dele ser mais modularizado que o modelo TCP/IP, eke possui mais 
protocolos específicos. Esses protocolos descrevem oque deve ser feito, mas não a forma
que deve ser, também descrevendo a forma de interação de cada camada com as camadas 
vizinhas.

\newpage

\begin{center}
  \begin{tabular}{| m{3cm} | m{7cm} |}
  \hline
  \textbf{Camada do Modelo OSI} & \textbf{Descrição} \\
  \hline
  7 - Aplicação & Contém os protocolos usados para processo a processo comunicativo. \\
  \hline
  6 - Apresentação & Fornece uma representação comum dos dados transferidos entre serviços da camada de aplicativo. \\ 
  \hline
  5 - Sessão & A camada de sessão fornece serviços para camada de apresentação para organizar o diálogo e gerenciar intercâmbio de dados. \\
  \hline
  4 - Transporte & A camada de transporte define serviços para segmentar, transferir e remontar dados para comunicações individuais entre os dispositivos finais. \\
  \hline
  3 - Rede & Fornece serviços para troca de partes individuais de dados dos quadros entre dispositivos finais. \\
  \hline
  2 - Enlace de Dados & Descrevem métodos para troca de dados entre dispositivos comuns.\\
  \hline  
  1 - Físico & Descrevem as partes físicas e processuaus para manter e desativar as conexões físicas para transmissão de bits \\

 
\hline
\end{tabular}
\end{center}
% subsubsection Modelo referência OSI (end)

\subsubsection{Modelo de referência TCP/IP} % (fold)
\label{sec:Modelo de referência TCP/IP}
Ele diferente do modelo OSI, possui menos divisões, sendo somente 4 grade divisões. Ele
foi craido nos anos 70 e corresponde à estrutura de um conjunto específico de protocolos
 porque descreve as funções que ocorrem em cada camada por meio de protocolos delas. 

\begin{center}
\begin{tabular}{| m{3cm} | m{7cm} |}

  \hline
  \textbf{Camda do modelo TCP/IP} & \textbf{Descrição} \\
  \hline  
  4 - Aplicação & Representa dados para o usuário, além do controle de codificação e do diálogo \\
  \hline
  3 - Transporte & Permite a comunicação entre vparios dispositivos diferentes em redes distintas.\\
  \hline
  2 - Internet & Determina o melhor caminho pela rede\\
  \hline
  1 - Acesso à rede & Controla os dispositivos de hardware e os meios físicos que formam a rede.\\

  \hline
\end{tabular}
\end{center}
% subsubsection Modelo de referência TCP/IP (end)
% subsection Modelos de Referência (end)

\subsection{Encapsulamento de Dados} % (fold)
\label{sub:Encapsulamento de Dados}
\subsubsection{Segmentando Mensagens} % (fold)
\label{sec:Segmentando Mensagens}
Em teoria, uma únia comunicação poderia ser enviada através de uma rede por meio de um 
fluxo maciço de bits. No entanto, isso criaria problemas comunicativos entre os 
dispositivos conectados nessa rede que precisassem usar os mesmo canais comunicativos, 
resultando em atrasos consideráveis de comunicação. Por isso, as mensagens são 
dividias em pequenos pacotes que são mais fáceis de serem gerenciáveis pela rede. 
Segmentação é o processo de dividir um fluxo de dados em unidades menores pra 
transmissão. Ela é necessária porque as redes de dados usam o conjunto de protocolos 
TCP/IP para enviar dados em pacotes IP individuais. Isso promove duas coisas, 
\textbf{aumento de velocidade de transmissão de dados} e \textbf{aumento de eficiência}.

Esses dois elementos permitem a \textbf{multiplexação}, que é quando duas conversas 
diferentes são intercaladas e gerenciadas na rede ao mesmo tempo.
% subsubsection Segmentando Mensagens (end)

\subsubsection{Sequenciamento} % (fold)
\label{sec:Sequenciamento}
Na comunicação em rede, cada segmento de mensagem deve passar por um processo 
semelhante para garantir que chege ao destino correto e possa ser remontado no 
conteúdo da mensagem original. Para isso, temos que o sequenciamento de pacotes
é feito para que a ordem de entrega seja feita de maneira correta, e caso algum 
pacote seja perdido, pode ser identificado e recuperado.
% subsubsection Sequenciamento (end)
% subsection Encapsulamento de Dados (end)


\subsection{Acesso a dados} % (fold)
\label{sub: Acesso a dados}
As camada de rede e de enlace de ados são responsáveis por entregar os dados do 
dispositivo origem para o dispositivo de destino. Ois protocolos nas duas camadas 
contêm um endereço de origem e de destino, mas seus endereços têm finalidades 
diferentes:

\begin{itemize}
  \item \textbf{Endereços de origem e destino da camada de rede: } Responsável por entregar o pacote IP da origem original ao destino final.
  \item \textbf{Endereços de origem e destino na camada de enlace de dados: } Responsável por fornecer o quadro de enlace de dados de uma palca de interface de rede para outra na mesma rede.
\end{itemize}

\subsubsection{Endereço Lógico da Camada 3} % (fold)
\label{sec:Endereço Lógico da Camada 3}
Um endereço IP é o endereço lógico da camada de rede, usando para entregar o pacote IP 
da origem original ao destino final. O pacote Ip contém o endereço IP de origem e de 
destino do pacote. Ele também contém dentro de si, um endereço IP com duas partes, sendo elas: 

\begin{itemize}
  \item \textbf{Parte da rede (IPv4) ou Prefixo(IPv6): }A parte mais à esquerda do endereço que indica qual rede o endereço IP é membro, Todos os dispositivos na mesma rede terão a mesma parte da rede.
  \item \textbf{Parte do host (IPv4) ou ID da interface (IPv6): } A parte restante do endereço que identifica um dispositivo específico na rede. É uma parte exclusiva para cada dispositivo ou na interface na rede.
\end{itemize}



% subsubsection Endereço Lógico da Camada 3 (end)

% subsection Acesso a dados (end)








  


\end{document}
