\documentclass[12pt, a4paper]{paper}

\title{Técnicas básicas de grafos}
\author{João Vitor Rezende Moura}
\begin{document}

\section{Infográfico} % (fold)
\label{sec:Infográfico}
\subsection{Coloração sequencial de grafos} % (fold)
\label{sub:Coloração sequencial de grafos}
A convenção do uso de cores em grafos se origina no seu uso para a coloração de países 
em mapas. Em representações matemáticas e computacionais, é comum usar os primeiros
números inteiros positivos ou não negativos como as cores. Em geral pode-se usar 
qualquer conjunto finito de cores. A natureza do problema depende da quantidade das 
cores, e não da sua natureza.
% subsection Coloração sequencial de grafos (end)
% section Infográfico (end)

\section{Livro: Ordenação de elementos em grafos} % (fold)
\label{sec:Livro: Ordenação de elementos em grafos}
Uma aresta direcionada de um grafo é um par ordenado de vértices (u, v) representando
uma conexão assimétrica de \textit{u} a \textit{v}. As arestas normais podem ter seu 
comportamente repetido por meio de arestas bidirecionais, ou de duas arestas direcionais,
cada uma delas ligando pro o outro elemento.

Um grafo direcionado e sem ciclos é chamado de \textbf{grafo acíclico direcionado}, e 
podem também ser chamados de DAGs. Eles possuem muitos recursos que os tornam mais 
simples que os grafos. Para começar, sempre são dotados de vértices conhecidos como 
fontes e sumidouros. Uma \textbf{fonte} é um vértice sem aresta de entrada; um 
\textbf{sumidouro} é um vértice sem saída.

\subsection{Ordenação topológica} % (fold)
\label{sub:Ordenação topológica}
A ordenação topológica funciona como uma ordenação natural das DAGs. E ela dá suporte a 
um recurso visual de disposição de vértices e arestas do grafo, Para isso, é preciso 
listar todas as fontes da esquerda para a direita e representar as arestas direcionadas 
de G entre essas fontes. Na verdade, qualquer lsita de vértices é uma ordenação topológica se todas as arestas estiverem direcionadas até o fim da lista.

\subsubsection{Algoritmo de ordenação Topológica} % (fold)
\label{sec:Algoritmo de ordenação Topológica}
Um dos algoritmos comumente empregados para a obtenção de uma ordem topológica de um 
grafo faz uso de um procedimento de varredura, que o algoritmo de 
\textbf{busca em profundidade.}
Esse algoritmo possui uma estrutura de dados auxiliar para armazenar os vértices na 
ordem em que a visitação de suas vizinhanças é finalizada. 
\newline

\texttt{\textbf{Algoritmo 1: Busca em profundidade (BP)}}
\hrule
%Corrigir essa parte destacada %%%%%%%%%%%%%%%%%%%%%%%%%%%%%%%%%%%%%%%%%%%%%
\texttt{
  para cada v ∈ V(G) faça 
    v.status ← NOVO
  fim para 

  tempo ← 0 
  pilhaVertices ← ∅
  para cada v ∈ V(G) faça 
    se v.status = NOVO entao
      visitaBP(v, listaVertices)
  fim se,para 
  retorna listaVertices
}
  %%%%%%%%%%%%%%%%%%%%%%%%%%%5




% subsubsection Algoritmo de ordenação Topológica (end)
% subsection Ordenação topológica (end)
\subsection{Coloraçao em grafos} % (fold)
\label{sub:Coloraçao em grafos}

% subsection Coloraçao em grafos (end)

% section Livro: Ordenação de elementos em grafos (end)


















\end{document}
