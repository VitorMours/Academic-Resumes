\documentclass[12pt, a4paper]{paper}

\author{João Vitor Rezende Moura}
\institution{UNIT- Universidade Tiradentes}

\begin{document}
\section{Infográfico: Etapas do algoritmo de fluxo máximo} % (fold)
\label{sec:Infográfico: Etapas do algoritmo de fluxo máximo}

\subsection{Fluxo definido em zero} % (fold)
\label{sub:Fluxo definido em zero}
A primeira etapa é a definição do fluxo ao longo de cada aresta para 0, considerando-se 
que não há nenhum fluxo nesse momento. Posteriormente, usa-se o algoritmo de busca de 
caminhos, como o busca em largura, ou profundidade para encontrar um caminho P de s para 
t que tenha capacidade disponível.
% subsection Fluxo definido em zero (end)

\subsection{Capacidade do caminho} % (fold)
\label{sub:Capacidade do caminho}
Para encontrar a capacidade desse caminho, é preciso olhar para todas as arestas e no 
caminho subtrair seu fluxo de corrente de sua capacidade. Em seguida aumenta-se o fluxo 
por emio das arestas no caminho P pelo valor de caminho(P). Repeteindo o processo até 
que não tenha mais caminhos do início ao fim disponíveis.
% subsection Capacidade do caminho (end)

\subsection{Grafo residual} % (fold)
\label{sub:Grafo residual}
Basicamente, pega-se o grafo existente e atualizam-se as capacidades de todas as arestas
regulares para a capacidade restante atual. Em seguida, adicionam-se bordas de volta 
indicando a quantidade de fluxo atualmente passando por essa borda.
% subsection Grafo residual (end)

\subsection{Definição do algoritmo} % (fold)
\label{sub:Definição do algoritmo}
É a etapa em que se define o algoritmo a ser usado. O algoritmo mais comum referenciado
é o algoritmo de \textbf{Ford-Fulkerson}, que também é conhecido como algoritmo dos 
pseudocaminhos aumentadores.
% subsection Definição do algoritmo (end)
% section Infográfico: Etapas do algoritmo de fluxo máximo (end)
\newpage

\section{Livro: Teoria dos grafos e o fluxo máximo de redes} % (fold)
\label{sec:Livro: Teoria dos grafos e o fluxo máximo de redes}
Temos dentro da teoria dos grafos, que o \textbf{peso} é uma variável associada a 
arestas, e que a partir dela podemos determinar a capacidade possível existentes de uma 
determinada aresta. Já o fluxo máximo é definido como a quantidade máxima de fluxo que 
a rede permite fluir da fonte para o coletor. 

\subsection{Problema do fluxo máximo em redes} % (fold)
\label{sub:Problema do fluxo máximo em redes}
Um fluxo é definido por um grafo direconado envolvendo uma fonte e um consumidor e 
vários outros nós conectados por arestas. Cada borda tem uma capacidade individual, assim
o limite máximo de fluxo que a borda permite. 
Um \textbf{dígrafo} com um valor inteiro \textit{c} definida em seu conjunto de arcos é 
chamada de rede capacitada. Nessa rede temos a fonte, destino, e os intermediários. Um 
fluxo na rede é uma função de valor inteiro definida em seu conjunto de arcos tal que 
a soma dos fluxos ao longo de todos os aros direcionados ao vértice \textit{i} é a 
entrada em \textit{i}, e a soma dos fluxos ao longo de todos os arcos direcionados ao 
vértice \textit{i} é a saída de \textit{i}.




% subsection Problema do fluxo máximo em redes (end)
% section Livro: Teoria dos grafos e o fluxo máximo de redes (end)






\end{document}
