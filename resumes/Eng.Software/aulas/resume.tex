\documentclass[12pt a4paper]{paper}

\usepackage{array}


\begin{document}
  
\section{Engenharia de Requisitos} % (fold)
\label{sec:Engenharia de Requisitos}
\subsection{Requisitos de Sistema} % (fold)
\label{sub:Requisitos de Sistema}
Requisitos são características do sistema ou a descrição de algo que o sistema é capaz 
de realizar para atingir seus objetivos.
Eles também podem ser entendidos como uma condição ou capacidade que deve ser 
alcançada ou estar presente em um sistema para satisfazer um contrato, padrão, 
especificação ou outro documento formalmente imposto.

Os maiores desafios no desenvolvimento de sistemas, é a compreensão do domínio do 
problema, comunicação efetiva com os usuários reais e a evolução contínua dos 
requisitos do sistema.
% subsection Requisitos de Sistema (end)
% section Engenharia de Requisitos (end)

\section{Diagramas de caso de uso e UML} % (fold)
\label{sec:Diagramas de caso de uso e UML}
A UML é uma linguagem visual para especificar, cosntruir e documentar os artefatos do 
sistema. Um apoio ao processo de desenvolvimento de forma visual.
Com a adoção do do paradigma OOP, surgiram diversas formas de modelar um sistema, com 
isso, a UML surgiu. Ela permite que cada elemento gráfico possua uma sintaxe e uma 
semãntica que define o significado do elemento e para que ele deve ser usado. 

A criação do UML e da linguagem de modelagem envolve a criação de diversos documentos, 
sendo alguns deles. 

\begin{itemize}
  \item Documentos textuais e gráficos 
  \item Documentos de artefato de software 
  \item Artefatos que compõem as visões do sistema 
\end{itemize}

Os modelos de caso de uso representam as funcionalidades de forma a ser observada 
externamente o sistema, e a forma como ele interage com si mesmo. 


\subsection{Caso de Uso} % (fold)
\label{sub:Caso de Uso}
O modelo de caso de uso possibilita a apresentação compreensível que pode ser utilizada 
para aprimorar a comunicação, especialmente entre projetistas da aplicação e clientes. 
Eles são úteis para fases posteriores do ciclo de vida, ajudando na udentificação dos objetos, desenvolvimento de planos de teste e documentação.

Os casos de uso representam quem faz o que com o sistema, sem considerar o comportamento
o comportamento interno do sistema, e esses comportamentos podem ser descritos de 
diversas maneiras, como por meio de descrições: Narrativas; Contínuas; Numeradas e 
Cliente/Sistema. 

Também pode ser tomado em conta o nível de detalhamento a ser utilizado na descrição 
de um Caso de Uso, podendo ser sucinta, ou expandida. 

Os casos de uso possuem três elementos principais dentro deles, que são os casos de uso, 
os atores e os relacionamentos entre os elementos anteriores.
% subsection Caso de Uso (end)

\subsection{Atores} % (fold)
\label{sub:Atores}
Os atores são elementos externos que interagem com o sistema, e eles podem ser 
categorizados de diversas formas, como por exemplo: Pessoas; Organizações; Outros 
sistemas e Equipamentos. Um ator corresponde a um papel representado em relação ao 
sistema.

Os atores podem ser tanto \textbf{primários}, que é quando eles inicam a sequência de 
interações de um caso de uso, como podem ser \textbf{secundários}, que é quando eles 
operam, mantêm ou auxiliam na utilização do sistema.
% subsection Atores (end)

\subsection{Relacionamentos} % (fold)
\label{sub:Relacionamentos}
Os relacionamentos dento da UML podem ser divididos em: 

\begin{itemize}
  \item Comunicação ou associação 
  \item Inclusão 
  \item Extensão 
  \item Generalização
\end{itemize}

A \textbf{inclusão} ocorre quando dois ou mais casos de uso incluem uma sequência de 
interações comum, esta sequência comum pode ser descrita e um outro caso de uso. 

A \textbf{extensão} por outro lado, utiliza a ideia de deferentes sequências de 
interações podem ser inseridas em um caso de uso. Por exemplo, uma determinada ação 
como editar um documento, pode ter além dessa atribuição, a possibilidade de corrigir 
a ortografia automaticamente, isso diz respeito a extensão dos elementos serem 
extendidos.

A \textbf{generalização} é quando o reuso é extremamente evidente, e ele pertime que um 
caso de uso herde características de outro caso de uso, porém mais genérico.
% subsection Relacionamentos (end)

\subsection{Documentação dos atores dos casos de uso} % (fold)
\label{sub:Documentação dos atores dos casos de uso}
Uma breve descrição para cada ator deve ser adicionada ao modelo de caso de uso. O nome 
do ator deve lembrar o papel desempenhado pelo sistema.
% subsection Documentação dos atores dos casos de uso (end)

\subsection{Regras de Negócio} % (fold)
\label{sub:Regras de Negócio}
São políticas , condições ou restrições que devem ser consideradas na execução dos 
processos existentes em uma organização. Eles descrevem a maneira pela qual a 
organização funciona. Estas regras são identificadas e documentadas no modelo de 
regras do negócio
% subsection Regras de Negócio (end)
% section Diagramas de caso de uso (end)

\section{Diagrama de Classes } % (fold)
\label{sec:Diagrama de Classes }
Os diagramas de classes nos ajudam a representar coias no mundo real ou imaginário que 
podemos facilmente identificar. E isso é feito por meio do paradigma de orientação a 
objetos. Um \textbf{Objeto} pode ser entendido como um elemento, que possui atributos 
e propriedades dentro de si. 
As classes são representaads por retângulos com três divisões dentro de si, onde cada 
uma dessa divisões possui uma finalidade dentro de si. Sendo o nome, atributos e 
métodos respectivamente.

Os \textbf{Relacionamentos} possibilitam compartilhar informações  e colaborar com a 
execução dos processos do sistema. Os relacionamentos podem ser de cinco tipos 
diferentes, sendo eles: 

\begin{itemize}
  \item Associação 
  \item Agregação 
  \item Composição 
  \item Especialização 
  \item Dependência
\end{itemize}

\subsection{Multiplicidade} % (fold)
\label{sub:Multiplicidade}
A multiplicidade diz respeito a quantidade de elementos que devem ser relacionados nas 
formas de relacionamentos presentes dentro da engenharia de software, e possuem os 
seguintes tipos com os seguintes significados:

\begin{center}
  \begin{tabular}{| m{2.5cm} | m{8cm} |}
  \hline
  \textbf{Multiplicidade} & \textbf{Significado} \\
  \hline
  0..1 & No mínimo zero e no máximo um, devem se relacionar com no máximo um elemento \\
  \hline
  1..1 & Um e somente um \\
  \hline
  0..* & No mímimo nenhum e no máximo muitos \\
  \hline
  * & Muitos \\
  \hline
  1..* & No mínimo um no máximo muitos \\
  \hline
  3..5 & No mínimo 3 no máximo 5 \\
\hline
\end{tabular}
\end{center}
% subsection Multiplicidade (end)



\subsection{Associação} % (fold)
\label{sub:Associação}
Descreve um conjunto de vínculos entre elementos de modelo. Possui um relacionamento 
estrutural que especifica objetos de um item conectados a objetos de outro item. A 
associação ainda pode ser divida em \textbf{binária} e \textbf{unária}, que é quando 
há duas classes envolvidas na associação de forma direta de uma para outra, e quando 
há um relacionamento de classe consigo mesma, respectivamente. 

A simbologia é representada por uma linha preta completa, sem nenhum tipo de elemento 
em seu final.
% subsection Associação (end)

\subsection{Agregação} % (fold)
\label{sub:Agregação}
É um tipo especial de associação que tenta demonstrar que as informações de um objeto 
como um todo precisam ser complementadas pelas informações contidas em um objeto parte. 
A existência do objeto-parte faz sentido mesmo não existindo o objeto-todo.

Sua notação no diagrama de classes, é feita por meio de uma linha que possui um 
diamante vazio na ponta do elemento que é o objeto-parte
% subsection Agregação (end)

\subsection{Composição} % (fold)
\label{sub:Composição}
É uma variação ad agregação e considerada mais "forte", e tem a ideia de que o 
objeto-parte não pode existir sem o objeto-todo, portanto, se ele for destruído, o 
objeto-parte também será destruído. 

Tem como representação gráfica, uma linha com o diamante na ponta do elemento que é 
o objeto-parte, mas que dessa vez possui em si, o preenchimento do losango.
% subsection Composição (end)

\subsection{Dependência} % (fold)
\label{sub:Dependência}
 Indica um grau de dependência entre duas classes, e difere da associação porque 
 é uma conexão entre as classes de forma temporária, e não permanente.
% subsection Dependência (end)
% section Diagrama de Classes  (%#Endregion
 \end{document}
