% Project Model and Packages
\documentclass[12pt, a4paper]{paper}
\usepackage{babel}
\usepackage{graphicx}
\usepackage[margin=2.5cm]{geometry}

% University logo Setting

% Title, format and first informations
\title{Engenharia de Software \\
  \hfill\includegraphics[height=3cm]{images/universidade.png}
  \vspace{-3cm}
}
\subtitle{Pela ótica de Roger S. Pressman e Bruce R. Maxim}
\author{João Vitor Rezende Moura - 1221184773}
\institution{UNIT - Universidade Tiradentes}
\begin{document}

% Parte de cima que tras informações sobre o resumo, título, palavras chaves e afins.

\maketitle

\hrule
  \begin{abstract}
  \textit{colocar aqui um resumo de engenharia de software quando terminar o livro}
  \end{abstract}

  \begin{keywords}Engenharia; Software; Métodos; Processos; Ferramentas; Cosntrução; Padrões\end{keywords}
\hrule

\section{Software e Engenharia de Software}
Dentro da produção de qualquer software, existem padrões que devem ser seguidos, por serem referência na forma como funcionam, ou por serem a melhor escolha em determinados casos, seja desde a forma como a construção do software é pensado, até mesmo a forma como ele se comporta e se organiza durante o seu funcionamento. Essa estrutura e padrões que foram anteriormente citados, são os processos, conjuntos de métodos e gama de ferramentas que constituem a \textit{engenharia de software}.

\subsection{A natureza do software}
Nos dias atuais, o software tem um papel que é ao mesmo tempo, duplo em seu uso, e fundamental na sua presença dentro da sociedade contemporânea. Ele pode ser visto tanto como um produto, como um veículo capaz de distribuir um produto por meio dele.

\begin{itemize} 
  \item \textbf{Produto: } Fornece o potencial computacional representado pelo hardware, ou por uma rede de computadores que pode ser acessada via hardware. Seja localizada em um dispositivo móvel, ou em um comptuador de nesa, na nuvem ou em um \textit{mainframe}, o software transforma informações.
  
  \item \textbf{Veículo: } O softwre atua como base para o controle do computador, comunicação de informações e a criação e controle de outros programas.
\end{itemize}

Devido à sua natureza, podemos ver --- principalmente nos dias atuais e na forma como as relações se comportam --- que o software distribui um dos produtos mais importantes da atualida, a \textit{informação}. Ela permite transformar dados pessoais em informações úteis conforme o contexto da necessidade do seu uso. O papel do software mudou no decorrer dos últimos 60 anos. O aperfeiçoamento no desempenho do hardware, mudanças de arquiteturas compputacionais e aumento na capacidade de armazenamento e de processamento resultou em sistemas computacionais mais sofisticados e complexos.\par 
Atualmente, uma indústria de \textit{software} tornou-se domunante nas economias do mundo industrializado. Equipes de especialistadas concentram-se uma parte tecnológica necessária para distribuir uma aplicação complexa.

\subsubsection{Definição de Software}
A maior parte dos profissionais e muitos outros integrantes gerais não possuem um entendimento preciso e técnico do que é um software. O mesmo pode ser descrito como:

\vspace{0.5cm}

``
 Software consiste em: (1) instruções que, quando executadas fornecem características, funções e desempenhos desejados; (2) estruturas de dados que possibilitam aos programas manipular informações adequadamente; e (3) informação descritiva, tanto na forma impressa quanto na forma virtual, descrevendo o uso dos programas
``

\vspace{0.5cm}

É importante que examinemos as características que compõem um software, e que o tornam diferente das outras coisas. Ele é mais um elemento do sistema lógico do que físico, Portanto, o software tem um característica que o torna consideravelmente diferente do hardware, e é fundamental para o seu entendimento como construção \textit{ele não se desgasta}. Apesar disso, temos outro fator que é particular do software, que apesar dele não se desgastar, ele pode apresentar defeitos na sua idealização e construção, entretanto, esses defeitos são corrigidos, novos aparecem posteriormente, provando que, apesar de não se desgastar, o software deteriora.
\par 
Outro aspecto é que cada software indica um erro no projeto ou no processo pelo qual o projeto foi traduzido em código, Portanto, a manutenção do software implicam em complexidade consideravelmente maior do que a manutençaõ de hardware, devido à necessidade de entendimento lógico do sistema, se não como um todo, como uma parte dele.

\subsubsection{Domínios de aplicação do software}
Atualmente, existem sete grande categorias de software existentes, e cada uma delas apresenta seus desafios de construção, manutenção, e de abstração para os engenheiros de software, sendo os tipos, os seguintes:
\begin{itemize}

  \item \textbf{Software de Sistema: } Conjunto de programas feitos para atender outros programas. Alguns deles processam estruturas informacionais complexas --- porém determinadas --- e retornam seu processamento ao sistema.
  \item \textbf{Software de Aplicação: }Programas independentes que solucionam uma necessidade específica do negócio. Aplicações nessa área processam dados comerciais ou técnicos de uma forma que facilite operações comerciais ou tomadas de decisões estratégicas.
  \item \textbf{Software de Engenharia \textit{ou} Científico: } Ampla variedade de programas que possibilitam cálculos em massa, abrangem várias áreas, como: Astronomia; Vulcanologia; Análise de estresse automotivo e etc\dots
  \item \textbf{Software Embarcado: } Residente num produto ou sistema e utilizado para implementar e controlar características e funções para o usuário e para o próprio sistema. Executa funções limitadas e específicas, ou fornece função significativa e capacidade de controle.
  \item \textbf{Software para Linha de Produtos: } Composto por componentes reutilizáveis e projetado para prover capacidades específicas de utilização por muitos clientes diferentes. Concentra-se maios no mercado hermético e limitado.
  \item \textbf{Aplicações Web \textit{ou} Aplicações Móveis: } Categoria de softwre voltado às redes abrage uma ampla variedade de aplicações, como aplicações desktop, comptução em nuvem, e muitos outros.
  \item \textbf{Software de Inteligência Artificial: } Faz uso de heurísticas para solucionar problemas complexos que não são passíveis de computação ou de análise direta. Áreas que incluem essas aplicações são: Robótica; Sistemas de tomada de decisão; Reconecimento de padrões; Aprendizado de máquina e outros.
\end{itemize}

\subsubsection{Software legado}
Programas mais antigos, são softwares que possuem tecnologias defasadas, ou até mesmo 
sem suporte mais, e essas tecnologias são comuns dentro do mundo do software, e eles são 
comumente chamados de \textbf{\textit{software legado}}. Esses softwares muitas vezes 
possuem consigo, o problema de baixa qualidade, entretanto, pela natureza desse 
software, o mais recomendado é justamente não modificar seus elementos e fazê-lo 
somente quando necessário, seja para atender novas necessidades, consertar problemas. 
Ou quando o mesmo não é mais utilizável da forma o qual é presente.


\subsection{Definição da Disciplina} % (fold)
\label{sub:Definição da Disciplina}







% subsection Definição da Disciplina (end)

\end{document}
