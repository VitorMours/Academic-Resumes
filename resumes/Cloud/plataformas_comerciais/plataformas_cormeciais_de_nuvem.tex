\documentclass[12pt, a4paper]{paper}
\usepackage{babel}
\usepackage{graphicx}
\usepackage[margin=2.5cm]{geometry}
%\hfuzz=5.002pt 

\title{Plataformas comerciais de nuvem\\
  \hfill%\includegraphics[height=3cm]{../images/universidade.png}\vspace{-3cm}
}
\subtitle{Um estudo sobre as plataformas de nuvem e seus funcionamentos dentro do mercado de TI.}
\author{João Vitor Rezende Moura - 1221184773}
\institution{UNIT - Universidade Tiradentes}

\begin{document}
\maketitle

\hrule
\begin{abstract}
  abstract
\end{abstract}
\vspace{-0.6cm} % vspace 
\begin{keywords}
  keywords
\end{keywords}
\hrule

\section{Apresentação} % (fold)
\label{sec:Apresentação}
As plataformar comerciais de serviços na nuvem oferecem uma quantidade de serviços 
que vão desde a hospedagem, até o processamento de serviços de alto desempenho, 
passando até mesmo pela internet das coisas.
% section Apresentação (end)

\section{Infográfico} % (fold)
\label{sec:Infográfico}
Atualmente existem diversas plataformas comerciais de computação em nuvem, entretan- 
to, possuem 4 que dominam esse mercado, sendo elas:

\paragraph{Amazon Web Services:} % (fold)
\label{par:Amazon Web Services:}
Essa plataforma tem origem na Amazon, e ela proporcionou que a mesma se reiventasse. 
A varejista passo a ser uma das maiores empresas de tecnologia do planeta 
graças aos seus sevirços de nuvem disponbilizados na AWS. 
Eles possuem diversos serviços \textit{SaaS} --- indo desde armazenamento e backup até 
containers, blockchain e IoT --- e diversos outros tipos de serviços que são 
comercializados. Além disso, a AWS oferece ofertas que são sempre gratuitas e outras 
com períodos de teste gratuito para avaliação, antes de necessariamente iniciar a 
cobrança. 
% paragraph Amazon Web Services (end)

\paragraph{Google Cloud:} % (fold)
\label{par:Google Cloud:}
Essa organização sempre teve seus esforços concentrados nas aplicações pela Internet, com produtos focados nos usuários finais. Com a evolução natural do conceito de nuvem, também passou a comercializar sua infraesutrutra robusta na nuvem com serviços como \textit{IaaS, PaaS e SaaS}. 
O diferencial ofertado é o app, chamado de \textbf{App Engine}, que permite criar 
aplicativos móveis para a web com escalabilidade automática.
% paragraph Google Cloud:  (end)

\paragraph{IBM Cloud:} % (fold)
\label{par:IBM Cloud:}
Se trata de uma paltafor de serviços Paas, com enfoque no desenvolvimento de soluções,
com amplo suporte a diversas linguagens de programação e serviços web. 
Como diferencial, a IBM fornece sua inteligência artificial, o \textbf{Watson}, que 
possui dentro de si o uso de muitas APIs que facilitam o desenvolvimento de soluções 
inteligentes.
% paragraph IBM Cloud:  (end)

\paragraph{Microsoft Azure:} % (fold)
\label{par:Microsoft Azure:}
É a detentora da maioria dos usuários de computadores pessoais e proffisionais há 
décadas. Depois de não ter conseguido implancar o Windows phone, decidiu focar nos 
serviços de nuvem, pois proporcionava uma maior chance de se manter no topo. 
Uma das principais caracterísiticas é o incentivo ao uso da sua maior paltaforma SaaS, 
que é o \textit{\textbf{Office 365}}. 
% paragraph Microsoft Azure:  (end)
% section Infográfico (end)

\section{Livro - Aplicação da Internet das Coisas} % (fold)
\label{sec:Livro - Aplicação da Internet das Coisas}

\subsection{Introdução às paltaformas comerciais de nuvem} % (fold)
\label{sub:Introdução às paltaformas comerciais de nuvem}

  \subsubsection{Amazon Web Services } % (fold)
  \label{sec:Amazon Web Services }
  A AWS oferece um amplo conjunto de produtos globais em nuvem, e esses serviços 
  ajudam as empresas a ganhar agilidade, baixar os custos da área de TI e crescer em 
  escala. As maiores empresas e startups mais promissores do mercado confiam na 
  plataforma para uma grande variedade de cargas de trabalho. A nuvem da AWS fornece 
  um total de 69 zonas de disponibilidade em 22 regiões geográficas em todo o mundo, 
  além de ter planos divulgados para mais 13 zonas de disponibilidade e mais quatro 
  regiões.
  % subsubsection Amazon Web Services  (end)

  \subsubsection{IBM Cloud} % (fold)
  \label{sec:IBM Cloud}
  Nem todos os servidores em nuvem são criados da mesma forma. As ofertas de nuvem
  pode maximizar os lucros e aperfeiçoar as cargas de trabalho, entretanto, a 
  IBM Cloud utiliza padrões que podem ser configurados por hora ou até mesmo por mês.
  A plataforma combina os modelos de plataforma e de infraesutrutra, sendo 
  \textit{PaaS e IaaS} respectivamente, fornecendo com isso uma experiência integrada.
  A paltaforma suporta e adapta-se tanto a pequenas equipes de desenvolvimento quando 
  a grandes organizações e empresas corpora- 
  tivas. Soluções nessas plataformas iniciam rapidamente, sendo executadas de forma 
  confiável.
  % subsubsection IBM Cloud (end)

  \subsubsection{Microsoft Azure} % (fold)
  \label{sec:Microsoft Azure}
  A plataforma da microsoft também oferece computação em nuvem, e outros serviços, e 
  tende a possuir muitas vantagens comparada a servidores locais. Não possui muito 
  destaque operacional e de sistema, diferente dos outros sistemas, mas é muito 
  útil para usuários do sistema o qual o mesmo funciona.
  % subsubsection Microsoft Azure (end)

  \subsubsection{Google Cloud} % (fold)
  \label{sec:Google Cloud}
  Conssite num conjunto de aplicações que pode ser acessível tanto diretamente na 
  nuvem, como por meio de ferramentas, como o Google Cloud Shell CLI. Além disso, 
  não necessitar de fazer instalações e o uso de licenças, ajuda muito. A grande 
  característica da plataforma da google são o \textbf{Google Kuberbetes Engine}, 
  \textbf{Google Compute Engine} e \textbf{Google App Engine}.
  % subsubsection Google Cloud (end)
% subsection Introdução às paltaformas comerciais de nuvem (end)

\subsection{Serviços disponíveis nas nuvens comerciais} % (fold)
\label{sub:Serviços disponíveis nas nuvens comerciais}
Os três tipos de serviços padrões que podem ser disponibilizados nas provedoras de 
computação em nuvem, são infraestrutura, uma paltaforma ou um software, todos eles 
como um serviço, criando assim as categorias de \textit{IaaS, PaaS e SaaS}. Entretanto, 
os três possuem variações nas suas estruturas, de forma que deve-se atentar ao mesmo. 

O modelo \textbf{SaaS} permite que uma empresa publique seu software e ofereça amplo
acesso a seus usuários por meio de um navegador Web. Servidores de suíte, 
como o Microsoft Office 365 ou aplicativos como o Salesforce, fonecem aos usuários 
acesso isntantâneo a documentos e arquivos sem a necessidade de instalar, gerenciar e 
armazenar aplicativos e dados em seus dispositivos pessoais. A característica mais 
marcante desse tipo de serviço, é os usuários não possuirem a necessidade de arcar 
com atualizações caras e licenças, basendo o serviço em assinaturas. 

O modelo \textbf{PaaS} significa que uma empresa usa \textit{software e hardware} 
fornecidos por uma provedor de soluções cloud para criar e implantar seu próprio 
conjunto de serviços. Exemplos famosos dessas plataformas são a AWS e Heroku. Uma das 
vantangens é utilizar as funções de teste e compilação do PaaS para se comunicar e 
entregar com eficiência novos desenvolvimentos de produção em uma rede nacional ou 
internacional estendida. Dessa forma, é possível evitar investimentos caros.

O último modelo é o \textbf{IaaS}, e ele é conhecido como computação de utilitários, 
pois ele permite que os usuários obtenham acesso a servidores, armazenamento e 
\textit{hardware} por meio da nuvem. Esse é o modelo mais flexível, permitindo que os 
usuários personalizem seus produtos ao máximo, pois possuem acesso direto a servidores 
externos, não requerendo investimentos de capital interno em hardwares específicos.
% subsection Serviços disponíveis nas nuvens comerciais (end)
\end{document}

