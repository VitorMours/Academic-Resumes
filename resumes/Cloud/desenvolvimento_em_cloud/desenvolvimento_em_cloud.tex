\documentclass[12pt, a4paper]{paper}
\usepackage{babel}
\usepackage{graphicx}
\usepackage[margin=2.5cm]{geometry}
%\hfuzz=5.002pt 

\title{title \\
  %\hfill\includegraphics[height=3cm]{../images/universidade.png} \vspace{-3cm}
}
\subtitle{subtitle}
\author{full name - badge number}
\institution{University}

\begin{document}
\maketitle

\hrule
\begin{abstract}
  abstract
\end{abstract}
\vspace{-0.6cm} % vspace 
\begin{keywords}
  keywords
\end{keywords}
\hrule

\section{Infográfico} % (fold)
\label{sec:Infográfico}
A computação em nuvem surgiu para auxiliar tanto os provedores quanto os clientes. Cada modelo de serviço tem suas particularidades e aplicabilidade no contexto a Internet. 

O modelo de \textbf{PaaS} suporta um conjunto de determinadas linguagens fornecidas 
e tecnologias fornecidas pelas empresas, permitindo que as empresas aluguem serviços 
para comprar ou criar suas próprias aplicações. O acesso e gerenciamento podem ser 
feitos pelo navegador de um cliente proprietário da empresa de computação em nuvem ou de
algum ambiente integrado. 

Por outro lado, o \textbf{SaaS} é um sofware oferecido e o cliente somete o utiliza. 
Esse software provavelmente foir desenvolvido utilizando uma PaaS e agora está pronto 
e isntalado na infraestrutura da nuvem. Na visão do cleitne, não é possível gerenciar 
nem conhecer detalhes físicos de onde o sfotware é gerenciado.
% section Infográfico (end)

\section{Livro} % (fold)
\label{sec:Livro}

\subsection{Modelos de Progamação em Cloud Computing} % (fold)
\label{sub:Modelos de Progamação em Cloud Computing}
A computação em nuvem atraiu a atenção tanto da academia quanto da indústria, e agora engloba vários segmentos, afetando diretamente a sociedade em geral. Ela oferece serviços aos seus clientes de várias maneiras. 

Toda e qualquer aplicação envolve uma estratégia de desenvolvimento específica. Em geral, todos os processos sã ocaracterizados pelo uso de um modelo de programação específico escolhido com base no tipo de aplicação a ser desenvolvida.

Ao longo dos anos, no que diz respeito às técnicas de programação em geral, há uma 
evolução contínua, relacionada a novos paradigms tecnológicos. Cada etapa evolutiva 
é caracterpizada por certas alterações na metodologia existente e, emcada nível, uma 
nova funcionalidade é adicionada, oque facilita o trabalho dos programadores. 

\subsubsection{Modelos de programação estendidos para a nuvem } % (fold)
\label{sec:Modelos de programação estendidos para a nuvem }
Nem todos os métodos de programação podem ser migrados para a nuvem. Esitem várias 
propriedades da nuvem que a tornam uma tecnologia diferente das outras. Portanto, certas
alterações precisam ser feitas para solucionar os problemas. A principal questão dos 
modelos aceitos de serem usados na nuvem, são aqueles que tem como possibilidade, a 
grande escalabilidade dos serviços e da forma do seu uso.

A primeira é a MapReduce, ele oferece suporte aos desenvolvedores para escrever as 
aplicações que podem processar grandes quantidades de dados não estruturados em 
paralelo em um ambiente de processamento distribuído. Ele se baseia na ideia de que 
processamento paralelo de aplicações com uso intenso de dados. Essas instâncias são 
executadas em paralelo para processar o conjunto de dados. Depois de concluídos os 
trabalhos de computação, a segunda fase começa. Nele os resultados intermediários 
produzidos por instâncias individuais são mesclados para produzir o resultado final. 

Outro modelo que se originou desse foi o CGL-MapReduce, pois ele faz uso do streaming 
de dados em vez de um sistema de arquivos
% subsubsection Modelos de programação estendidos para a nuvem  (end)

\subsection{Novos modelos de programação propostos para a nuvem} % (fold)
\label{sub:Novos modelos de programação propostos para a nuvem}





% subsection Novos modelos de programação propostos para a nuvem (end)






% section Livro (end)










\end{document}

