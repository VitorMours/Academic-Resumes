\documentclass[12pt, a4paper]{paper}
\usepackage{graphicx}

\title{Estrutura de Dados e Seus Algoritmos}
\subtitle{pela ótica de Lilian Markenzon e Jayme Luiz Szwarcfiter\\
  \hfill\includegraphics[height=2.5cm]{images/universidade.png}
  \vspace{-2cm}
}
\author{João Vitor Rezende Moura}
\institution{UNIT - Universidade Tiradentes}

\begin{document}

\section{Introdução} % (fold)
\label{sec:Introdução}
% section Introdução (end)

\section{Lista Lineares} % (fold)
\label{sec:Lista Lineares}
  \subsection{Introdução} % (fold)
  \label{sub:Introdução}
  Dentre as estruturas de dados não primitivas, as listas lienares são as de manipulação mais simples. Uma lista linear agrupa isnformações referentes a um conjunto de elementos que, de alguma forma, se relacionam entre si, podendo ser ińumeros os tipos de dados que podem ser descritos por listas lineares.\par
  Uma lista linear também pode ser entendida como uma tabela ou um vetor, ela é composta por um conjunto de \textit{n $>=$ 0} nós tais que suas propriedades estruturam decorrem unicamente da posição relativa dos nós dentro da sequência linear. Os comportamentos mais comuns das listas --- e que na verdade estão presente na maioria das estruturas de dados --- são de remoção, adição e busca. Entretanto, existem tipos de listas com variaççoes específicas desses comportamentos que podem ser entendidas como estruturas de dados únicas pela forma como se comportam, e como seu uso único é muito particular, porém útil, não somente na computação. Algumas dessas "listas especiais" são:
\begin{itemize}
  \item \textbf{Fila:} Quando a inserção dos elementos é feita de um lado da lista, mas a remoção é feita no outro lado da lista, e cada uma dessas ações é exclusivamente executada no seu lado correspondente, seguindo o padrão \textit{FIFO}.
  \item \textbf{Pilha:} A inserção e a deleção de elementos são feitos no mesmo lado da filha, e são exclusivamente feitos desse lado, proporcionando assim o uso do princípio \textit{LIFO}
  \item \textbf{Deque:} É uma abreviação de \textit{double ended queue} e ela tem como comportamento, a possibilidade de adição e deleção de elemento dos dos lados da fila, e apenas nas extremidades, assim como as estruturas especiais anteriormente citadas. 
\end{itemize}
  O armazenamento dos elementos presentes dentro de uma lsia linear pode ser classificado de acordo com a posição relativa na memória de dois nós consecutivos na lista. Quando o armazenamento é sempre contínuo, podemos dizer que é uma \textit{alicação sequencial de memória}, por outro lado, quando isso não é feito, podemos dizer que temos uma \textit{alocação encadeada}. A escolha de um tipo ou do outro depende essencialmente das operações que são executadas sobre a lista, do número de listas envolvidas e como as características particulares dessas listas.
  % subsection Introdução (end)
  \subsection{Alocação Sequêncial} % (fold)
  \label{sub:Alocação Sequêncial}
  A maneira mais simples de se manter uma lista linear na memória de um computador é colocar seus nós em posições contínuas. Como a implementação da alocação sequencial em linguagems de alto nível é geralmente realizada com reserva prévia de memória, a inserção e remoção de nós não ocorrem de fato. Normalmente são usadas variáveis para "definir" os limites da memória, oque nos faz concluir que essa alocação pode ser considerada \textit{estática}.\par
  Ele é particularmente atraente no caso de filas e pilhas já que suas opera-
  ções básicas podem ser implementadas de forma bastante eficiente. Entretanto, é muito oneroso computacionalmente falando, essas operações em quesito de memória, por isso, um estudo da sua necessidade deve ser feito antes da sua implementação.
  % subsection Alocação Sequêncial (end)
  \subsection{Pilhas e Filas} % (fold)
  \label{sub:Pilhas e Filas}
  O armazenamento sequencial de listas é empregado quando as estruturas ao longo do tempo, sofrem poucas remoções e inserções. Em casos particulares de listas, esse armazenamento é também empregado. Nesse caso, a situação favorável é aquela em que inserções e remoções não acarretam movimentação de nós, oque ocorre se os elementos a serem inseridos e removidos estão em posições especiais, com oa primeira ou a última posição.
  % subsection Pilhas e Filas (end) 
  \subsection{Alocação Encadeada} % (fold)
  \label{sub:Alocação Encadeada}
  A implementação de operações realizadas em listas com alocações sequenciais  é fraca, principalmente quando há utilizaçã oconcomitante de mais de duas listas, pois torna a gerência de memória mais complexa, com isso, justifica-se o uso de alocação encadeada, que também pode ser chamada de \textit{alocação dinãmica}. Os nós de uma lista são alocados e desalocados conforme a necessidade, e com isso, eles se encontram aleatoriamente dispostos na memória e são interligados por ponteiros, que indicam a posição do próximo elemento da tabela. É necessário o acréscimo de um campo a cada nó, justamente oque indica o endereco do próximo nó da lista. Apesar dessa forma necessitar de mais memória por utilizar mais um campo informacional, é mais conveninente quando o problema inclui o tratamento de mais de uma lista, isso se aplica tanto à gerência do armazenamento quanto às operações propriamente ditas envolvidas.
  % subsection Alocação Encadeada (end)
  \subsection{Lista Lineares em Alocação Encadeada} % (fold)
  \label{sub:Lista Lineares em Alocação Encadeada}


    \subsubsection{Listas Simplesmente Encadeadas} % (fold)
    \label{sec:Listas Simplesmente Encadeadas}
    Qualquer estrutura que seja armazenada em alocação encadeada requer o uso de 
    um tipo de ponteiro que indique o endereço de seu primeiro nó. O percurso da lista
    é feito a partir dos ponteiros, seguindo consecutivamente os endereços fornecidos 
    a cada um deles, até chegar ao objetivo. 

    Esse tipo de estrutura, pode apresentar erros, como por exemplo, considerar durante 
    um algoritmo de busca, que o nó cabeça não possui um ponteiro indicado que ele é 
    o primeiro, necessitando dentro da busca, casos de teste especiais para verificar 
    que ele é o primeiro elemento. Isso pode ser resolvido com uma modificação da 
    estrutura que provê uma classificação especial aos primeiros e último nós, oque 
    facilita a a busca deles dentro do sistema. 


    \subsubsection{Pilhas e Filas} % (fold)
    \label{sec:Pilhas e Filas}
    Como em casos particulares, modificações são necessárias para a eficiência da 
    implementação de operações em pilhas e filas. No caso das \textbf{pilhas}, não é 
    necessário a presença do nó cabeça, pois temos que o fim da inserção e o fim 
    da início da exclusão é sempre feito a partir do elemento que está no topo 
    da pilha, facilitando assim o algoritmo. No caso das \textbf{filas}, temos duas 
    variáveis que apontam para o primeiro e o último elementos inseridos dentro da 
    estrutura, em que são respecitivamente a saída e a entrada da fila.
    % subsubsection Pilhas e Filas (end)

    \subsubsection{Lista Duplamente Encadeada} % (fold)
    \label{sec:Lista Duplamente Encadeada}
    A lista duplamente encadeada possui comportamento similar ao da lista simplesmente 
    encadeada, com a única diferença de que um ponteiro exta é adicionado, e que esse 
    mesmo indica o elemento anterior o qual acabou de ser citado e/ou analisado. 

    Outro fator que deve ser analisado, e que é parecido com o da lista simplesmente 
    encadeada, é a necessidade da presença de uma variável especial para indicar o 
    início da lista, nesse caso a necessidade de indicar o último elemento da lista 
    também se faz presente, evitando assim buscas infinitas e circulares.
    
    \paragraph{Aplicação: Ordenação Topológica} % (fold)
    \label{par:Aplicação: Ordenação Topológica}
    A ordenação topológica é um problema muito presente dentro da matéria de Teoria 
    dos Grafos, e esse tipo de algoritmo pode ser resolvido com o uso de uma lista 
    duplamente encadeada. Sua importância se deve ao seu potencial de todas as vezes 
    que o problema abordado envolve uma ordem parcial --- que é entendida como um 
    conjunto S que é a relação entre os objetos de S, temos o entendimento da seguinte
    maneira:

    %
    % colocar a figura 2.13, e fazer notações matemáticas e explicar simbologia
    %
    
    % paragraph Aplicação: Ordenação Topológica (end)
    % subsubsection Listas Simplesmente Encadeadas (end)
  % subsection Lista Lineares em Alocação Encadeada (end)  
% section Lista Lineares (end)

\section{Árvores} % (fold)
\label{sec:Árvores}
\subsection{Definição e Representação Básica} % (fold)
\label{sub:Definição e Representação Básica}
Temos que uma árvore enraizada, que é mais comumente conhecida como somente árvore, 
é um conjunto finito de elementos denominados de \textit{nós} ou \textit{vértices}, tais
que 
\begin{itemize}
  \item A árvore é dita vazia na ausência de elementos.
  \item Existe um nó espécial chamado raiz. Os restantes constituem um único conjunto vazio, ou são divididos em $m \geq 1$ conjuntos disjuntos não vazios.

\end{itemize}

% subsection Definição e Representação Básica (end)








% section Árvores (end)
\end{document}
